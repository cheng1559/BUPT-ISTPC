\documentclass [11pt, a4paper, oneside] {article}

\usepackage {CJKutf8}
\usepackage [T2A] {fontenc}
\usepackage [utf8] {inputenc}
\usepackage [english, russian] {babel}
\usepackage {amsmath}
\usepackage {amssymb}
\usepackage {olymp}
\usepackage {comment}
\usepackage {epigraph}
\usepackage {expdlist}
\usepackage {graphicx}
\usepackage {multirow}
\usepackage {siunitx}
\usepackage {ulem}
%\usepackage {hyperref}
\usepackage {import}
\usepackage {ifpdf}
\ifpdf
  \DeclareGraphicsRule{*}{mps}{*}{}
\fi

\begin {document}
\begin {CJK}{UTF8}{gbsn}

\contest
{程序设计周赛-10}%
{}%
{}%

\binoppenalty=10000
\relpenalty=10000

\renewcommand{\t}{\texttt}


\graphicspath{{../../problems/r10-path-huaguang/statements/chinese/}}
  \def\ProblemIndex{A}
\import{../../problems/r10-path-huaguang/statements/chinese/}{./problem.tex}
\graphicspath{{../../problems/r10-fontaine-whale/statements/chinese/}}
  \def\ProblemIndex{B}
\import{../../problems/r10-fontaine-whale/statements/chinese/}{./problem.tex}
\graphicspath{{../../problems/r10-sacred-sakura-cleansing-ritual/statements/chinese/}}
  \def\ProblemIndex{C}
\import{../../problems/r10-sacred-sakura-cleansing-ritual/statements/chinese/}{./problem.tex}
\graphicspath{{../../problems/r10-cat-tail/statements/chinese/}}
  \def\ProblemIndex{D}
\import{../../problems/r10-cat-tail/statements/chinese/}{./problem.tex}
\graphicspath{{../../problems/r10-shortest-path/statements/chinese/}}
  \def\ProblemIndex{E}
\import{../../problems/r10-shortest-path/statements/chinese/}{./problem.tex}

\end {CJK}
\end {document}
