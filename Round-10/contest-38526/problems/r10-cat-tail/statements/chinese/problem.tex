\begin{problem}{蒙德的卡牌}{standard input}{standard output}{1 second}{256 megabytes}

七圣召唤是风靡提瓦特的卡牌游戏,这个游戏中每个角色牌有三种技能,在每回合的开始,玩家通过投掷获取一定数量的骰子,每个骰子可能是七种元素之一,也可能是无色骰子,玩家可以消耗骰子来发动一些技能。

蒙德的猫尾酒馆经常举行热斗模式的七圣召唤对决,这种模式下一般会有许多离谱的规则,比如减少技能所需骰子,所有骰子均为无色等等。

现在我们考虑一种简单的热斗模式对决,所有骰子均为无色,元素战技和普通攻击仅需$1$骰子,元素爆发仅需$2$骰子,且不再需要充能,场上只有一位角色。

为了避免浪费,我们总是希望用完所有骰子,你需要求出用完所有骰子的方案数。

具体地说,你一开始有$k$个骰子,每次你可以进行以下三个操作,直到剩余$0$个骰子为止:
\begin{itemize}
\item 普通攻击,消耗$1$骰子。
\item 元素战技,消耗$1$骰子。
\item 元素爆发,消耗$2$骰子。
\end{itemize}

两种操作方案被认为是不同的当且仅当它们总的操作次数不同,或者存在一个$i$使得两种操作的第$i$步操作不同。

由于这个数可能特别大,你只需要输出对$998244353$取模的结果。

\begin{center}
  \includegraphics[scale=0.15]{cattail.jpg} \\
  \small{猫尾酒馆}
\end{center}



\InputFile
一个整数$k$,一开始的骰子数量。($1 \le k \le 10^{15}$)

\OutputFile
一个整数,用完所有骰子的方案数,对$998244353$取模。

\Examples

\begin{example}
\exmpfile{example.01}{example.01.a}%
\exmpfile{example.02}{example.02.a}%
\exmpfile{example.03}{example.03.a}%
\end{example}

\Note
一些选手可能不知道如何计算$x^n$($n$是自然数),这里给出一种算法:

注意到:
$$
x^0 = 1\\
x^n = x^{n \bmod 2} \cdot (x^2)^{\lfloor n / 2 \rfloor}
$$
在编程中实际上将递归展开成循环,有算法:

$$
\begin{aligned}
& POWER(x, n):\\
& \hspace{2em}   r = 1\\
& \hspace{2em}   \boldsymbol{\operatorname{while }}\; n > 0:\\
& \hspace{4em}       \boldsymbol{\operatorname{if }}\; n \bmod 2 == 1:\\
& \hspace{6em}           r = r \times x\\
& \hspace{4em}       x = x \times x\\
& \hspace{4em}       n = \lfloor n / 2 \rfloor\\
& \hspace{2em}   \boldsymbol{\operatorname{return }}\; r
\end{aligned}
$$

时间复杂度: $O(\lg n)$

\end{problem}

